\documentclass{article}
% 这里是使用的一些包
\usepackage{amsmath, amsthm, amssymb, amsfonts}
\usepackage{thmtools}
\usepackage{graphicx}
\usepackage{setspace}
\usepackage{geometry}
\usepackage{float}
\usepackage{hyperref}
\usepackage[utf8]{inputenc}
\usepackage[english]{babel}
\usepackage{framed}
\usepackage[dvipsnames]{xcolor}
\usepackage{tcolorbox}
% 定义的一些颜色
\colorlet{LightGray}{White!90!Periwinkle}
\colorlet{LightOrange}{Orange!16}
\colorlet{LightGreen}{Green!15}

\newcommand{\HRule}[1]{\rule{\linewidth}{#1}}
% 声明的一些定理以及一些东西
\declaretheoremstyle[name=Theorem,]{thmsty}
\declaretheorem[style=thmsty]{theorem}
\tcolorboxenvironment{theorem}{title=Theorem,colframe=red!75!black,colback=LightGray}

\declaretheoremstyle[name=Proposition,]{prosty}
\declaretheorem[style=prosty]{proposition}
\tcolorboxenvironment{proposition}{colback=LightOrange}

\declaretheoremstyle[name=Principle,]{prcpsty}
\declaretheorem[style=prcpsty]{principle}
\tcolorboxenvironment{principle}{colback=LightGreen}

\declaretheoremstyle[name=Example,]{examsty}
\declaretheorem[style=examsty]{example}
\tcolorboxenvironment{example}{title=Example,colframe=red!55!black!50!yellow!50!,colback=red!50!green!20!blue!30}

\declaretheoremstyle[name=Remark,]{remarksty}
\declaretheorem[style=remarksty]{remark}
\tcolorboxenvironment{remark}{title=Remark,colframe=red!75!black,colback=red!40!green!30!blue!30}
% 边缘的一些东西
\setstretch{1.2}
\geometry{
    textheight=9in,
    textwidth=5.5in,
    top=1in,
    headheight=12pt,
    headsep=25pt,
    footskip=30pt
}

